\documentclass[a4paper,10pt]{report}
\usepackage[utf8]{inputenc}
\usepackage[toc]{glossaries}
\usepackage[xindy]{imakeidx}
\makeindex
\makeglossary
% Title Page
\title{The MRX  Project}
\author{Daniel Kettle}


\begin{document}
\newglossaryentry{bore}
{
  name={bore},
  description={The diameter of an engine's cylinder},
  plural={bores}
}
\newglossaryentry{stroke}
{
  name={stroke},
  description={The total length that a piston travels in an engine's cylinder in one direction},
  plural={strokes}
}
\newacronym[longplural=Horsepowers]{hpLabel}{HP}{Horsepower}
\maketitle


\begin{abstract}
\end{abstract}

\chapter{Pre-Reading}
\begin{itemize}
 \item MR2 Mk2 Roof Racks 
 \begin{verbatim}
  http://www.mr2oc.com/showthread.php?t=396411
 \end{verbatim}

\end{itemize}


\chapter{Changelog}
% List of changes from MR2 Configuration to MRX

\chapter{Exterior}

\section{Body}
\paragraph*{}This section can be applicable to both interior and exterior as it affects both, but mainly exterior. The problem here is that the likelyhood of finding a Mk1 Toyota MR2 in very good condition or better is highly unlikely and expensive. With the advent of 3D metal printing, a better situation \textit{could} be to find a good body and re-print the body panels. To go even further, I could find a completely different car body, strip it down to the body and custom build it from there. From then on, finding a car with a body similar to that MR2 style is less trivial than ripping apart an actual model. For instance, Volvo bodies are boxy, but too long. If the added length is acceptable, that could be an option. A Fiero body could also work, but its age might work against it. There are not many cheap 2 seater or small new FWDs being produced and newer means more expensive. The Mk1 MR2 was built on an FWD body, so that isn't a problem; most FWDs are being produced as roundy looking aerodynamic blobs of ugly; I will have to deal with the roof since it determines part of the shape.

\paragraph*{}Another point I would like to make is that if the body has a rear windshield and is a 4 seater, I would remove the glass windshield, install a metal grate-like sloping component where the glass used to be, seal off part of the back compartment with heat shield and find a windshield that fits above the heat shield and seals the top. Obviously, a 4 seater would involve much more work.



\section{Lights}
\paragraph*{}All light bulbs should be replaced with efficient LEDs that are bright enough and legal to change over to. While decreasing electrical requirements very slightly, it will help in case of burnt out or non-functional bulbs due to redundancy.

\section{Vents}
\paragraph*{}The MR2 Mk1 has a few vents placed in various locations on the vehicle that allow for heat dissipation and air intake. Increasing the number of vents on the vehicle allows for up to 3 special cases in the MRX:
\begin{itemize}
 \item Cooling for radiators
 \item Airflow (increase aerodynamics, releasing pressure areas, preventing buildup of hot air)
 \item Air Intakes for the engine or intercooler
\end{itemize}

\section{Doors}
\paragraph*{}Doors should be custom made, or stripped out to save weight. I do not require windows to be rolled down, but if that is the case I should develop a moonroof that can easily be removed in case of an accident.
\paragraph*{}Doors will be modified as scissor doors, opening upwards instead of outwards. This is to prevent dings when opening, as well as to look flashy and sporty. If windows do not open, then opening the door is similar in effect.


\section{Dual Side Air Intakes}
\paragraph*{}The original MR2 Mk1 has one air intake on the passanger side of the vehicle with a fan and attached radiator / cooling system. This is used to cool the engine bay and coolant systems. For the ~100HP 4-AGE engine, this is adequate and there have been no heat issues during the lifetime of my MR2 ownership. However, replacing an engine with up to 2.4x more horsepower and lesser fuel economy may raise overall heat in the bay. This heat issue may be solved with a radiator above the engine similar to the Supercharged MR2 Mk1 version, where radiator is right below the engine lid (that has vents carved into it) to allow heat dissipation. 
\paragraph*{}The point of having a secondary Air Intake on the Drivers-side of the body (where the fuel cap is located) is to enable either dual cooling systems or to enable a turbo air intake. The bay is rather small, so while a rotary engine would allow for more space, it still leaves little room for a turbo or additional cooling system. If cooling can be satisfied with an undercarriage cooling system and/or top engine radiator, or if the front radiator (where most conventional car radiators are located) was connected to the engine bay, twin turbos could be a very real possibility. If that's the case, I would have two smaller, responsive turbos installed to reduce turbo lag as much as possible while boosting engine performance.
\paragraph*{See Engine Section ``Ram Air Intake'' }



\chapter{Interior}
\section{Dashboard}
\paragraph*{}This will be custom made, but using the same odometer/spedometer box that comes with the MR2. 
\subsection{Audio}
\paragraph*{}Audio is provided through the custom in-car audio entertainment system using embedded ARM and a custom UI I would like to design, but until that happens, it can be a Debian or Gentoo install.
\paragraph*{}A 7 or 10 inch touch screen will either go where the current audio and fan/heating controls are, or will replace the glovebox with an optional tiny LED readout where the audio deck used to reside. This is because I may replace the fan/heat control deck. See below.
\section{Fan and Heat Deck}
\paragraph*{}My experience with the removal of this deck from my car has been very difficult. Something is getting snagged or is bolted in from behind the deck, not allowing me to remove it. The Haynes manual says after removing the nut that is from the fan dial, it should be removeable. It is not. I will replace the fan deck with one of my own, with each dial or slider attached to one component dedicated to that function. 
\paragraph{Example}: The cable that controls the fan will be attached to a dial that controls the fan; the temperature slider will be attached to a knob that controls the temperature or a slider panel. Each component will fit into a plastic faceplate with cables long enough to easily remove and install. In the event of a control requiring a mechanical interface to the car function (not connected with a cable, but a wire or bar), I may have to create a component to convert that mechanical signal into an electrical one.
\section{Lights}
\paragraph*{}Strips of LEDs covered by a thin film (translucent, to prevent glare and blinding light) that can be dimmed should be placed in the cab: 
\begin{itemize}
 \item Around the moonroof
 \item On the doors
 \item Around switches and components, such as engine latch, arm rest, and above the footwells facing down to the floor.
 \item Strips on both sides of the gearbox facing the doors
\end{itemize}


\chapter{Performance}
\section{Engine}
\subsection{Rotory Bearings and Eccentric Shaft} 
\paragraph*{}Any shaft component that is lighter and stronger than a default OEM Eccentric Shaft is one that potentially can increase overall rpm and longevity in an engine, particularly in high pressure turbo engines.
\subsection{Intercooler} 
\paragraph*{}Because of the inefficiencies in a rotory engine regarding total fuel combustion and heat, placing an intercooler between the engine and airfilter (or replacing airfilter altogether if cooler provides a filter) could increase fuel economy and environmental factors that are more of an issue with the engine. \textit{RESEARCH}
\paragraph*{}Placement of the intercooler if on behalf on the engine would go in one of the dual side air intakes. If used as an air conditioner for the cab, it could be modified to work with the radiator or be placed in a hood scoop position.
\paragraph*{Final Note} Intercoolers without turbos may decrease the total amount of horsepower due to energy requirements.
\subsection{Turbo Configuration} 
\paragraph*{}Turbos, especially in a rotory engine, are effective. However, turbo lag is one feature I despise. There are a few possible options, none of which are optimal. 
\begin{itemize}
 \item \textbf{Supercharged} - Less complex mechanism with no lag, uses more fuel and engine must be built well to handle additional horsepower, both from mechanical energy reduction from running the supercharger and from additional horsepower.
 \item \textbf{Small Turbo} - Great power at low rpms (a problem with rotary engines), but has limited use in the entire range. With high rpms being a selling point of the rotory engine, this negates benefit and engine becomes more complicated to maintain.
 \item \textbf{Large Turbo} - More lag and low power at low rpms. Unacceptable.
 \item \textbf{Twin Turbo Configuration} - With two turbos that split exhaust, this covers the entire range but as with turbos, will change dynamic of the car, still have some (minor) lag and much additional complexity.
 \item \textbf{Two Stage Turbo} - Two turbos in sequence, best option for using turbos but still has the issues of complexity, cost and heat.
 \item \textbf{Twin Charged} - Supercharger and Turbo in use, expensive and inefficient although power output and response would be astounding.
 \item \textbf{Naturally Aspirated} - While keeping low rpms fairly weak, the engine would be in a much lighter body, modified to be lighter and perhaps rev higher. Torque would be less of an issue and horsepower per tonne would be increased. Response would still be great, least amount of complexity (unless intercooler was connected) and turbos can always be added later. This is the best option until I have more experience in what I want.
\end{itemize}
\subsection{Ram Air Intake}
\paragraph*{}Ram Air Intakes are air intakes that use vehicle movement to cut and force air into an intake which allows for higher oxygen content due to air physics. Efficiency gains will be in single digit percentages as Ram Air Intakes do not change the atmospheric pressure in a great way below super sonic speeds.
\paragraph*{}These air intakes are not usable on carbarated engines unless the design accomodates the air intake.
\section{Bore}\Gls{bore} is not applicable here. \glspl{bore}. \Gls{stroke}

\chapter{Environmental Impact}


\printglossaries
\end{document}          
