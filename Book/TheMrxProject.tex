\documentclass[a4paper,10pt]{report}
\usepackage[utf8]{inputenc}

% Title Page
\title{}
\author{}


\begin{document}
\maketitle

\begin{abstract}
\end{abstract}

\chapter{Changelog}
% List of changes from MR2 Configuration to MRX
\chapter{Exterior}
\section{Dual Side Air Intakes}
\subsection{Function}
\paragraph*{}The original MR2 Mk1 has one air intake on the passanger side of the vehicle with a fan and attached radiator / cooling system. This is used to cool the engine bay and coolant systems. For the ~100HP 4-AGE engine, this is adequate and there have been no heat issues during the lifetime of my MR2 ownership. However, replacing an engine with up to 2.4x more horsepower and lesser fuel economy may raise overall heat in the bay. This heat issue may be solved with a radiator above the engine similar to the Supercharged MR2 Mk1 version, where radiator is right below the engine lid (that has vents carved into it) to allow heat dissipation. 

\paragraph*{}The point of having a secondary Air Intake on the Drivers-side of the body (where the fuel cap is located) is to enable either dual cooling systems or to enable a turbo air intake. The bay is rather small, so while a rotary engine would allow for more space, it still leaves little room for a turbo or additional cooling system. If cooling can be satisfied with an undercarriage cooling system and/or top engine radiator, or if the front radiator (where most conventional car radiators are located) was connected to the engine bay, twin turbos could be a very real possibility. If that's the case, I would have two smaller, responsive turbos installed to reduce turbo lag as much as possible while boosting engine performance.


\chapter{Interior}
\chapter{Performance}
\chapter{Environmental Impact}


\end{document}          
